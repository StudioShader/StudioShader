%% start of file `template.tex'.
%% Copyright 2006-2015 Xavier Danaux (xdanaux@gmail.com).
%
% This work may be distributed and/or modified under the
% conditions of the LaTeX Project Public License version 1.3c,
% available at http://www.latex-project.org/lppl/.


\documentclass[12pt,a4paper,sans]{moderncv}        % possible options include font size ('10pt', '11pt' and '12pt'), paper size ('a4paper', 'letterpaper', 'a5paper', 'legalpaper', 'executivepaper' and 'landscape') and font family ('sans' and 'roman')

% moderncv themes
\moderncvstyle{casual}                             % style options are 'casual' (default), 'classic', 'banking', 'oldstyle' and 'fancy'
\moderncvcolor{blue}                               % color options 'black', 'blue' (default), 'burgundy', 'green', 'grey', 'orange', 'purple' and 'red'
%\renewcommand{\familydefault}{\sfdefault}         % to set the default font; use '\sfdefault' for the default sans serif font, '\rmdefault' for the default roman one, or any tex font name
%\nopagenumbers{}                                  % uncomment to suppress automatic page numbering for CVs longer than one page

% character encoding
\usepackage[utf8]{inputenc}                       % if you are not using xelatex ou lualatex, replace by the encoding you are using
%\usepackage{CJKutf8}                              % if you need to use CJK to typeset your resume in Chinese, Japanese or Korean

% adjust the page margins
\usepackage[scale=0.75]{geometry}
%\setlength{\hintscolumnwidth}{3cm}                % if you want to change the width of the column with the dates
%\setlength{\makecvtitlenamewidth}{10cm}           % for the 'classic' style, if you want to force the width allocated to your name and avoid line breaks. be careful though, the length is normally calculated to avoid any overlap with your personal info; use this at your own typographical risks...

% personal data
\name{Ivan}{Ogloblin}                             % optional, remove / comment the line if not wanted
\firstname{Ivan} % Your first name
\lastname{Ogloblin} % Your last name
\address{Rua Barata ribeiro 194}{Rio de Janeiro}{ Brazil }% optional, remove / comment the line if not wanted; the "postcode city" and "country" arguments can be omitted or provided empty
\phone[mobile]{+55~(21)982760687}                   % optional, remove / comment 
\email{ivanogloblin2022@gmail.com}                               % optional, remove / comment the line if not wanted                         % optional, remove / comment the line if                            % optional, remove / comment the line if not wanted
\social[github]{StudioShader}                              % optional, remove / comment the line if not wanted
% optional, remove / comment the line if not wanted; '64pt' is the height the picture must be resized to, 0.4pt is the thickness of the frame around it (put it to 0pt for no frame) and 'picture' is the name of the picture file                               % optional, remove / comment the line if not wanted

% bibliography adjustements (only useful if you make citations in your resume, or print a list of publications using BibTeX)
%   to show numerical labels in the bibliography (default is to show no labels)
\makeatletter\renewcommand*{\bibliographyitemlabel}{\@biblabel{\arabic{enumiv}}}\makeatother
%   to redefine the bibliography heading string ("Publications")
%\renewcommand{\refname}{Articles}

% bibliography with mutiple entries
%\usepackage{multibib}
%\newcites{book,misc}{{Books},{Others}}
%----------------------------------------------------------------------------------
%            content
%----------------------------------------------------------------------------------
\begin{document}
	%\begin{CJK*}{UTF8}{gbsn}                          % to typeset your resume in Chinese using CJK
	\clearpage
	%-----       letter       ---------------------------------------------------------
	% recipient data
	%	\recipient{Support Programm at IMPA}{Company, Inc.\\123 somestreet\\some city}
	\recipient{Cover letter}{}
	\date{\today}
	\opening{Dear Sir/Madam,}
	\closing{Yours faithfully,}
	%	\enclosure[Attached]{curriculum vit\ae{}}          % use an optional argument to use a string other than "Enclosure", or redefine \enclname
	\makelettertitle
	I am so excited to apply for the internship position as a software engineer! I am in a second year of my Master's degree in mathematics and this opportunity is perfect for me, as I will be able to apply all the experience I've gained during my studies as well as gain new, important technical knowledge.
	
	My Bachelor degree provided me with everything and more to be competent in any sub-field of software development. My strong side is research, analysis and development of algorithms. During my studies I worked a lot with genetic algorithms and quantum algorithms. Though they may only be applicable in specific cases, I learned a lot about general approaches to the most complex problems in the industry.
	
	I have a real experience working in different companies to produce value for the product development. At Huawei I worked as a full stack developer and learned a lot about the full cycle of developing a service designed for a large number of users and short response time requirements. At Yandex I wrote tests and tracked bugs in production which helped me internalize the importance of carefully developing levers in a complex system with a large number of services.
	
	I am a goal oriented person and my goal is to become a senior software developer as soon as possible. And this internship is the perfect opportunity for me to gain more hands-on experience and understand more about industry needs.
	

	
	%	\makeletterclosing
	\title{}
	\address{Rua Barata Ribeiro 194}{Rio de Janeiro}{ Brazil }
	\email{ivanogloblin2022@gmail.com}
	%	\makecvtitle
	\makeletterclosing
	\name{}{}
	%\clearpage\end{CJK*}                              % if you are typesetting your resume in Chinese using CJK; the \clearpage is required for fancyhdr to work correctly with CJK, though it kills the page numbering by making \lastpage undefined
\end{document}


%% end of file `template.tex'.
