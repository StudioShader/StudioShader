%% start of file `template.tex'.
%% Copyright 2006-2015 Xavier Danaux (xdanaux@gmail.com).
%
% This work may be distributed and/or modified under the
% conditions of the LaTeX Project Public License version 1.3c,
% available at http://www.latex-project.org/lppl/.


\documentclass[11pt,a4paper,sans]{moderncv}        % possible options include font size ('10pt', '11pt' and '12pt'), paper size ('a4paper', 'letterpaper', 'a5paper', 'legalpaper', 'executivepaper' and 'landscape') and font family ('sans' and 'roman')

% moderncv themes
\moderncvstyle{casual}                             % style options are 'casual' (default), 'classic', 'banking', 'oldstyle' and 'fancy'
\moderncvcolor{blue}                               % color options 'black', 'blue' (default), 'burgundy', 'green', 'grey', 'orange', 'purple' and 'red'
%\renewcommand{\familydefault}{\sfdefault}         % to set the default font; use '\sfdefault' for the default sans serif font, '\rmdefault' for the default roman one, or any tex font name
%\nopagenumbers{}                                  % uncomment to suppress automatic page numbering for CVs longer than one page

% character encoding
\usepackage[utf8]{inputenc}                       % if you are not using xelatex ou lualatex, replace by the encoding you are using
%\usepackage{CJKutf8}                              % if you need to use CJK to typeset your resume in Chinese, Japanese or Korean

% adjust the page margins
\usepackage[scale=0.75]{geometry}
%\setlength{\hintscolumnwidth}{3cm}                % if you want to change the width of the column with the dates
%\setlength{\makecvtitlenamewidth}{10cm}           % for the 'classic' style, if you want to force the width allocated to your name and avoid line breaks. be careful though, the length is normally calculated to avoid any overlap with your personal info; use this at your own typographical risks...

% personal data
\name{Ivan}{Ogloblin}                             % optional, remove / comment the line if not wanted
\firstname{Ivan} % Your first name
\lastname{Ogloblin} % Your last name
\address{Novoizmailovsky prospect, 16k8}{Saint-Petersburg}{ Russia }% optional, remove / comment the line if not wanted; the "postcode city" and "country" arguments can be omitted or provided empty
\phone[mobile]{+7~(913)~923~87 12}                   % optional, remove / comment 
\email{studioshader2018@gmail.com}                               % optional, remove / comment the line if not wanted                         % optional, remove / comment the line if                            % optional, remove / comment the line if not wanted
\social[github]{StudioShader}                              % optional, remove / comment the line if not wanted
% optional, remove / comment the line if not wanted; '64pt' is the height the picture must be resized to, 0.4pt is the thickness of the frame around it (put it to 0pt for no frame) and 'picture' is the name of the picture file                               % optional, remove / comment the line if not wanted

% bibliography adjustements (only useful if you make citations in your resume, or print a list of publications using BibTeX)
%   to show numerical labels in the bibliography (default is to show no labels)
\makeatletter\renewcommand*{\bibliographyitemlabel}{\@biblabel{\arabic{enumiv}}}\makeatother
%   to redefine the bibliography heading string ("Publications")
%\renewcommand{\refname}{Articles}

% bibliography with mutiple entries
%\usepackage{multibib}
%\newcites{book,misc}{{Books},{Others}}
%----------------------------------------------------------------------------------
%            content
%----------------------------------------------------------------------------------
\begin{document}
	%\begin{CJK*}{UTF8}{gbsn}                          % to typeset your resume in Chinese using CJK
	\clearpage
	%-----       letter       ---------------------------------------------------------
	% recipient data
	%	\recipient{Support Programm at IMPA}{Company, Inc.\\123 somestreet\\some city}
	\recipient{Support Programm at IMPA}{}
	\date{June 05, 2022}
	\opening{Research and Study Intentions}
	\closing{Yours truly,}
	%	\enclosure[Attached]{curriculum vit\ae{}}          % use an optional argument to use a string other than "Enclosure", or redefine \enclname
	\makelettertitle
	
	Quantum computing is a very intriguing field. Ideally, I want to push our knowledge to bring quantum computing to the industrial sector. I see an abundance of opportunities to utilize quantum computations in chemistry simulations. However, to understand how we can push the boundaries of quantum computing, we need to understand our current limitations. Today's quantum processors are small and noisy. I want to study and understand how we can overcome the low number of qubits and the high exposure to noise.
	
	In my 5th semester work, I researched best practices for solving logistics problems on classical computers and current results for quantum computers. As a result, I developed a QUBO formulation for the traveling salesmen problem and managed to implement it on Qiskit framework. I showed how unstably the ideally simulated VQE and QAOA algorithms work for 12 qubit size QUBO problem. Even without any noise they could not find a feasible solution.
	
	In my last project I implemented the thermal relaxation noise model and used it to demonstrate how local Grover's operator can improve the quantum search algorithm by reducing its circuit depth and thus reducing the effect of noise on the outcome of the algorithm.
	
%	It was theorized that we can use the "reset" gate to decrease the depth of the circuit of the unit increment operator. I have shown that it is not possible to effectively use the "reset" gate to reduce the depth of any unitary transformation without using information from the intermediate measurement. However, in some cases "reset" can be used for partial destruction of entanglement, which can potentially lead to less noise exposure.
	
	I intend to continue my study of the effects of noise on quantum circuits. My current tasks are to find a description of the influence of a simple noise model on the quantum search algorithm in terms of density matrix and test the hypothesis that the "reset" in the Shor's algorithm affects its exposure to noise.
	
	The problem of high noise stands in the way of expanding the capabilities of quantum processors. That means that there is a need to use various optimizations: the local Grover operator, error-correcting codes, the use of additional qubits to reduce the depth of the circuit. My project has the potential of finding the most noise-resistant improvements to algorithms and new optimization techniques.
	
	Ultimately, I am interested in the fields of quantum cryptography, quantum chemical simulations and quantum finite automata. My idea is to find an overlap between the potential of quantum computers and the quantum nature of chemical simulations. I already have experience with optimization algorithms, such as VQE and QAOA, which can be used in chemistry simulations. This summer I will be participating in \href{https://qiskit.org/events/summer-school/}{\textcolor{blue}{2022 Qiskit Global Summer School: Quantum Simulations}} and hope to gain as much experience as I can.
	
	Currently, I have one last year left before receiving a Bachelor's degree from St. Petersburg State University. But I have no supervisor to help me write my dissertation. Ideally, I would like to apply to the Master's degree at IMPA and get my Bachelor's online, if such an option is possible. Or I could complete my last year of Bachelor's degree in Brazil, if you know the right program. As I plan to advance my career in the academical field, after a Master's degree I will be aiming for a Ph.D.
	
	%	\makeletterclosing
	\title{}
	\address{Novoizmailovsky prospect, 16k8}{Saint-Petersburg}{ Russia }
	\email{studioshader2018@gmail.com}
	%	\makecvtitle
	\makeletterclosing
	\name{}{}
	%\clearpage\end{CJK*}                              % if you are typesetting your resume in Chinese using CJK; the \clearpage is required for fancyhdr to work correctly with CJK, though it kills the page numbering by making \lastpage undefined
\end{document}


%% end of file `template.tex'.
