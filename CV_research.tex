\documentclass[11pt,a4paper,sans]{moderncv}        % possible options include font size ('10pt', '11pt' and '12pt'), paper size ('a4paper', 'letterpaper', 'a5paper', 'legalpaper', 'executivepaper' and 'landscape') and font family ('sans' and 'roman')
\usepackage{lmodern}
% moderncv themes
\moderncvstyle{banking}                            % style options are 'casual' (default), 'classic', 'oldstyle' and 'banking'
\moderncvcolor{blue}                              % color options 'blue' (default), 'orange', 'green', 'red', 'purple', 'grey' and 'black'
%\renewcommand{\familydefault}{\sfdefault}         % to set the default font; use '\sfdefault' for the default sans serif font, '\rmdefault' for the default roman one, or any tex font name
\nopagenumbers{}                                  % uncomment to suppress automatic page numbering for CVs longer than one page

% character encoding
\usepackage[utf8]{inputenc}
%\usepackage{hyperref}
%\usepackage{pdfpages}
% if you are not using xelatex ou lualatex, replace by the encoding you are using
%\usepackage{CJKutf8}                              % if you need to use CJK to typeset your resume in Chinese, Japanese or Korean
\usepackage{multicol}

\usepackage{xcolor}
% adjust the page margins
\usepackage[scale=0.9,top=1.5cm, bottom=0.5cm]{geometry}
% \usepackage[scale=0.75]{geometry}
%\setlength{\hintscolumnwidth}{3cm}                % if you want to change the width of the column with the dates
%\setlength{\makecvtitlenamewidth}{10cm}           % for the 'classic' style, if you want to force the width allocated to your name and avoid line breaks. be careful though, the length is normally calculated to avoid any overlap with your personal info; use this at your own typographical risks...
\usepackage{xpatch}
\xpatchcmd\cventry{,}{}{}{}

% personal data

\name{Ivan}{Ogloblin}                               % optional, remove / comment the line if not wanted
\firstname{Ivan} % Your first name
\lastname{Ogloblin} % Your last name

% All information in this block is optional, comment out any lines you don't need
\title{Curriculum Vitae}

% \address{70 Absolute Ave.}{L4Z 0A4 Mississauga}{Canada}% optional, remove / comment the line if not wanted; the "postcode city" and and "country" arguments can be omitted or provided empty
\vspace*{3mm}
% optional, remove / comment the line if not wanted
% \phone[fixed]{+2~(345)~678~901}                    % optional, remove / comment the line if not wanted
% \phone[fax]{+3~(456)~789~012}                      % optional, remove / comment the line if not wanted
%  \homepage{linkedin.com/in/jondoe}                         % optional, remove / comment the line if not wanted
%\social[linkedin]{AlyaNovikova}
% \extrainfo{additional information}                 % optional, remove / comment the line if not wanted
%photo[64pt][0.4pt]{picture}                       % optional, remove / comment the line if not wanted; '64pt' is the height the picture must be resized to, 0.4pt is the thickness of the frame around it (put it to 0pt for no frame) and 'picture' is the name of the picture file
% \quote{Some quote}                                 % optional, remove / comment the line if not wanted

% to show numerical labels in the bibliography (default is to show no labels); only useful if you make citations in your resume
%\makeatletter
%\renewcommand*{\bibliographyitemlabel}{\@biblabel{\arabic{enumiv}}}
%\makeatother
%\renewcommand*{\bibliographyitemlabel}{[\arabic{enumiv}]}% CONSIDER REPLACING THE ABOVE BY THIS

%----------------------------------------------------------------------------------
%           footer
%----------------------------------------------------------------------------------
% bibliography with mutiple entries
%\usepackage{multibib}
%\newcites{book,misc}{{Books},{Others}}
%----------------------------------------------------------------------------------
%            content
%----------------------------------------------------------------------------------
%\makecvfooter
\begin{document}
	%\begin{CJK*}{UTF8}{gbsn}                          % to typeset your resume in Chinese using CJK
	%-----       resume       ---------------------------------------------------------
	\vspace*{-1.05mm}
	\makecvtitle
	\vspace*{-10mm}
	
	\section{Education}
	\cventry{}{}{Bachelor of Science in Computer Science and Software Engineering}{Sept 2019 - July 2023}{\hspace*{-2.5 mm} Saint-Petersburg State University}{}
	\cventry{}{}{Master of Science in Mathematics}{Sept 2022 - April 2025}{\hspace*{-2.5 mm} Pontifical Catholic University of Rio de Janeiro}{}
	{}{Related Coursework:}
	\vspace{-1.0em}\begin{small}
		\begin{multicols}{4}
			\begin{itemize}
				\item C++
				\item Kotlin
				\item Python
				\item Haskell
				\item Scala
				\item Algorithms
				\item Parallel programming
				\item Math logic
				\item Machine learning
				\item Unix
				\item Operating system
				\item Algebra
				\item Mathematical Analysis
				\item Random Process Theory
				\item Discrete Mathematics
				\item Statistics
				\item C\#
				\item Data Bases
				\item Quantum Computing
				\item Quantum Information
				\item JavaScript
				\item HTML and CSS
				\item Networks
				\item Quantum mechanics 
			\end{itemize}
	\end{multicols}\end{small}
	
	\section{Work Experience}
	
	\cventry{}{}{QC Design consultant}{February 2024 - current\vspace{-1.0em}}{}{
		% Detailed achievements:%
		%\begin{itemize}
		Work in a startup company on developing software for simulating quantum computing models in Python. Worked on optimizing performance of the codespace simulator, which involved understanding of the algorithm and implementing optimizations mostly based on manipulating data structures. Currently work on the Xpauli simulator project to make our software the fastest in the world.
	}
	%\end{itemize}}
	\cventry{}{}{Huawei Assistant Engineer, Developer}{October 2021 - January 2022\vspace{-1.0em}}{}{
		% Detailed achievements:%
		%\begin{itemize}
		Worked on backend C\#/.netASP/EntityFramework/Autofac + frontend 3js/react/VR. Developed system of package communication with no delay, that alternates between http and signalR requests. \\Did research work on handwriting recognition using convolutional network under "Human Computer Interactions". Got familiar with CNN, RNN and LSTM structures.
		%\end{itemize}
	}
	\cventry{}{}{Yandex Developer Intern}{July - Sept 2021\vspace{-1.0em}}{}{
		% Detailed achievements:%
		%\begin{itemize}
		Worked in two teams on backend C++/Python/SQL. Developed support system for training scripts to work with an optimized structure for storing variable logs. Wrote tests for components that were used to prepare data for a neural network that makes recommendations. Got acquainted with the concepts of services and levers. Dove into the intricacies of communication between services and systems for transmitting information with errors for debugging.
	}
	
	\section{Developer Projects}
	
	\cventry{}{}{Strawberry fields composer}{2023\vspace{-1.0em}}{}{
		% Detailed achievements:%
		Created a website dedicated to simulation of linear and non-linear optics for quantum computation models. Used Django, bootstrap and Strawberry fields. \href{https://strawberryfields.ai/}{\textcolor{blue}{Strawberry fields}} is a base for research directions. Right now it is hosted on an external free service, please wait while it loads! 
		\href{https://strawberryfieldscomposer.onrender.com}{(\textcolor{blue}{website})} (\href{https://github.com/StudioShader/SF_Composer}{\textcolor{blue}{github}})
	}
	
	\cventry{}{}{Archiver}{ 2019 \vspace{-1.0em}}{}{
		% Detailed achievements:%
			 C++ Used Huffman algorithm in implementation for data compression and decompression \href{https://github.com/StudioShader/huffman-archiver}{(\textcolor{blue}{github})}}
	% \vspace{1.0em}
	
	%	\cventry{}{}{Multithreading Paint
		%	}
	%	{July 2018 \vspace{-1.0em}}{}{
		%		% Detailed achievements:%
		%		\begin{itemize} \textbf {
				%			\itemG\url{https://github.com/AlyaNovikova/Multithreading-Paint}
				%			\item Programm implemented in \textbf{Java} with the use of multithreading.
				%			\item Multithreading Paint is a drawing program that allows multiple users to sketch on the same canvas simultaneously.
				%	\end{itemize}}
		
		
		
		\cventry{}{}{Vacanter}{2019\vspace{-1.0em}}{}{
			% Detailed achievements:%
			The Vacanter is a mobile application for matching employers with potential employees. I provided database and backend system for the application using postgreSQL, python, Datagrip.
			\href{https://github.com/AndreyYurko/hhTinder}{(\textcolor{blue}{github})} 
		}
		
		\section{Earlier achievements}
		
		\cventry{}{}{ICPC}{2020\vspace{-1.0em}}{}{
			% Detailed achievements:%
			\begin{itemize}
				\item \href{https://github.com/StudioShader/StudioShader/blob/main/2019-Northwestern_Russia-PLACE.pdf}{41 Place, Northwestern Russia Regional Contest St.Petersburg, October 26, 2019}
				\item \href{https://github.com/StudioShader/StudioShader/blob/main/2020-Northwestern_Russia-PLACE.pdf}{Honorable Mention, Northwestern Russia Regional Contest St.Petersburg, 14 November, 2020}
				%			\item \href{https://diploma.rsr-olymp.ru/files/rsosh-diplomas-static/compiled-storage-2018/by-code/117292234832/color.pdf}{Top 174 out of 1404 in "Open olympiad in Mathematics" 2017}
				%			\item \href{https://diploma.rsr-olymp.ru/files/rsosh-diplomas-static/compiled-storage-2018/by-code/117272475400/color.pdf}{Top 109 out of 1103 in "Open olympiad in Physics" 2018}
		\end{itemize}}
		
%		\cventry{}{}{Open olympiad}{2018\vspace{-1.0em}}{}{
%			% Detailed achievements:%
%			\begin{itemize}
%				\item \href{https://diploma.rsr-olymp.ru/files/rsosh-diplomas-static/compiled-storage-2018/by-code/117272475400/color.pdf}{Top 60 out of 1100 in "Open olympiad in Mathematics" 2018 and 2016}
%				\item \href{https://diploma.rsr-olymp.ru/files/rsosh-diplomas-static/compiled-storage-2018/by-code/117292234832/color.pdf}{Top 174 out of 1404 in "Open olympiad in Mathematics" 2017}
%				\item \href{https://diploma.rsr-olymp.ru/files/rsosh-diplomas-static/compiled-storage-2018/by-code/117272475400/color.pdf}{Top 109 out of 1103 in "Open olympiad in Physics" 2018}
%		\end{itemize}}
		
%		\cventry{}{}{International scientific school conference "XVIII Kolmogorov Readings" }{2019\vspace{-1.0em}}{}{
%			% Detailed achievements:%
%			\begin{itemize}
%				\item I took \href{https://internat.msu.ru/media/uploads/2018/05/pobediteli-informatika-na-sajt.pdf}{\textcolor{blue}{third place}} in the discipline of computer science and mathematical modeling
%		\end{itemize}}
		
		\section{Programming skills}
		\begin{itemize}
			\item C++, Python, C\#, C, Java, JavaScript, HTML, CSS, Kotlin, Haskell, Scala, SQL, Lean
			\item ASPnet, EntityFramework, Microsoft SQL Express, React, three.js, postgreSQL, Django, Bootstrap
			\item Git, Linux, Unity3D, SVN, Blender(3d modeling), protobuff, Shiny, Docker
			
		\end{itemize}
		%\vspace{6.0em}
		\newpage
		\section{Academic experience}
	%	\cventry{}{}{Photonic quantum computer architecrute}{2023-2024\vspace{-1.0em}}{}{
		%	As a Master's thesis I am doing research on architecture of photonic quantum computers. Specifically I research fault-tolerant approaches using surface codes and fusions (FBQC) with photons encoding qubits in KLM protocol. During this work I wrote in detail how to construct a universal gate set in this setup and showed how to mitigate some type of errors and how to run full cycle of any quantum algorithm in this architecture. (text to be published)
	%	}
		\cventry{}{}{Simulation of photonic quantum computing}{2023\vspace{-1.0em}}{}{
			Developed web service dedicated to simulation of linear and non-linear optics for quantum computation models using Python and Django. Used \href{https://strawberryfields.ai/}{\textcolor{blue}{Strawberry fields}} as an underlying engine. As it has a lot of components for non-linear optics, the simulation can take a while. During this work I got familiar with basic quantum optics notations and basic quantum computation models using non-linear optics. Right now it is hosted on an external free service, please wait while it loads! 
			\href{https://strawberryfieldscomposer.onrender.com}{(\textcolor{blue}{website})} (\href{https://github.com/StudioShader/SF_Composer}{\textcolor{blue}{github}})
		}
		\cventry{}{}{Undergraduate Thesis}{2022-2023\vspace{-1.0em}}{}{
			As my thesis I did research on optimal schemes of entangling transformations in linear quantum optics using genetic algorithm. I wrote genetic algorithm using GPUs on Python (Pytorch) to search for new entangling schemes. New schemes were obtained for finding the maximum entangled state, as well as for implementing gates equivalent to CX. Although it was not possible to improve the probability of operation, it was hypothesized that in the schemes in the KLM protocol it is impossible to find a scheme that implements the transformation, which would not be a perfect entangler, or at least not equivalent to CX. You can see presentation in 
			\href{https://github.com/StudioShader/galopy/blob/master/slides(eng).pdf}{\textcolor{blue}{this repository}}.
		}
	\cventry{}{}{Study of the Effect of Noise on Efficient Quantum Search Algorithms}{2022\vspace{-1.0em}}{}{
		Semester project on the topic "Study of the Effect of Noise on Efficient Quantum Search Algorithms". In this project I was to implement improved quantum search algorithms for unstructured DB. They are based on the Grover's algorithm and are described in \href{https://doi.org/10.1007/s11128-021-03165-2}{\textcolor{blue}{this} article}. The results of testing algorithms for a problem of no more than five qubits are shown. My task was to dig further into the limits of quantum search algorithms. First, I implemented improved search algorithms with Qiskit. Secondly, I created an environment for testing algorithms with different noise models and different numbers of qubits. Finally, I explored the impact of noise on variations of the algorithm. In my experiments I used thermal relaxation noise model and coupling map from a real device: "Melbourne". As a result I understood how to run such experiments in order to obtain estimates of the noise parameters for feasible operation of the algorithm. You can find details in \href{https://github.com/StudioShader/QPSA/blob/main/presentation.pdf}{\textcolor{blue}{this presentation}} or in \href{https://github.com/StudioShader/QPSA}{\textcolor{blue}{this repository}}.
	}
		\cventry{}{}{Quantum Algorithms for VRP and VRPTW Problems}{2021\vspace{-1.0em}}{}{
			A semester project on the topic “Quantum Algorithms for VRP and VRPTW (Vehicle Routing Problem with Time Windows) Problems” with application to the real case problems of building the routes for drilling machines for oil production in collaboration with GazpromNeft. I was directly assigned the task of studying current best practices for solving logistics problems on classical computers. The next step was to study current results on solving this problem by quantum and quantum-inspired algorithms. I found a reduction of this problem to QUBO (quadratic unconstrained binary optimization) and a solution using quantum optimization algorithms such as VQE and QAOA. Then I was to develop a simple solver for the multi-traveling salesman problem for small-scale problems (toy problem, up to 7 qubit). It can run locally on a simulator with Qiskit.\\
			(unfortunately I cannot share any code because of the privacy regulations of GazpromNeft)}
		\cventry{}{}{Teacher Assistant}{2023\vspace{-1.0em}}{}{
			I worked as a Teacher Assistant creating homework and complementing course notes on the course "introduction to quantum computation", for prof. \href{https://scholar.google.com/citations?user=UnZl40AAAAAJ&hl=en}{\textcolor{blue}{Sergey Tikhomirov}}. Broke down Shor's algorithm into sub-tasks and learned to explain concepts from basic to HHL.
		}
%		\cventry{}{}{Courses}{2021-2022\vspace{-1.0em}}{}{
%			Almost all of these topics were understood by the book\\ \href{https://www.cambridge.org/ru/academic/subjects/physics/quantum-physics-quantum-information-and-quantum-computation/quantum-computation-and-quantum-information-10th-anniversary-edition?format=HB&isbn=9781107002173}{\textcolor{blue}{"Quantum Computation and Quantum Information"} Michael A. Nielsen, Isaac L. Chuang}
%			\begin{itemize}
%				\item Course on introduction to quantum computations: Grover's algorithm, Deutsch–Jozsa algorithm, quantum permutations, quantum Fourier Transform, quantum search, Q-RAM, Shor's algorithm.
%				\item Course on quantum information: density operator, noise in quantum systems, closeness of quantum states, quantum correction codes and their realization, classical and quantum entropy, bandwidth of quantum channels, transmission of quantum information over a noisy quantum channel, quantum cryptography.
%				\item Additional seminar with the `GazpromNeft` team: Phase estimation algorithm, QAOA algorithm, QAA algorithm, VQE algorithm, quantum search as quantum simulation, black box algorithm limits, speed up of NP-complete problems, quantum search optimality, quantum search in unstructured database, physical realization of quantum computer: harmonic oscillator, optical photon quantum computers, optical cavity quantum electrodynamics
%			\end{itemize}
%		}
		
		\cventry{}{}{Qiskit Global Summer School 2022 - Quantum Excellence}{2022\vspace{-1.0em}}{}{
			I participated and excelled at Qiskit Global Summer School 2022 which was dedicated to quantum simulations. The main task was to find new ways to simulate hamiltonian for a particular physical system. I successfully solved it and earned a \href{https://www.credly.com/badges/3304071b-2191-46fe-9de6-0b1cc019a06f/public_url}{\textcolor{blue}{badge on Credly}}.
		}
		\cventry{}{}{Quantum Computing and Quantum Information via NMR}{2022\vspace{-1.0em}}{}{
			I participated and excelled at the 6th Advanced School of Experimental Physics of CBPF (Brazilian Center for Physics Research) and earned a  \href{https://github.com/StudioShader/StudioShader/blob/main/Certificados-21.pdf}{\textcolor{blue}{certificate}}. I had experience operating a real NMR device and running experiments with encoding and entangling two qubits.
		}
		
		
		
		% \vspace{-1.0em}
		
		% \vspace{1.0em}
		
		\section{Languages}
		Russian (Native), English (Fluent), Portuguese (Speaking)
		
		\vspace*{\fill}
		\name{}{}
		\title{}
		\phone[mobile]{+49~015162630532}
		\address{Bismarckring, 64}{Ulm}{Germany}
		\email{ivanogloblin2022@gmail.com}
		\social[github]{StudioShader}   
		\makecvtitle
		%\section{Languages}
		%\begin{itemize}
		%	\item Russian (Native), English (Upper-Intermediate)
		%\end{itemize}
		%\makecvfoot
	\end{document}