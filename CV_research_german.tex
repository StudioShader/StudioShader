\documentclass[11pt,a4paper,sans]{moderncv}        % possible options include font size ('10pt', '11pt' and '12pt'), paper size ('a4paper', 'letterpaper', 'a5paper', 'legalpaper', 'executivepaper' and 'landscape') and font family ('sans' and 'roman')
\usepackage{lmodern}
% moderncv themes
\moderncvstyle{banking}                            % style options are 'casual' (default), 'classic', 'oldstyle' and 'banking'
\moderncvcolor{blue}                              % color options 'blue' (default), 'orange', 'green', 'red', 'purple', 'grey' and 'black'
%\renewcommand{\familydefault}{\sfdefault}         % to set the default font; use '\sfdefault' for the default sans serif font, '\rmdefault' for the default roman one, or any tex font name
\nopagenumbers{}                                  % uncomment to suppress automatic page numbering for CVs longer than one page

% character encoding
\usepackage[utf8]{inputenc}
%\usepackage{hyperref}
%\usepackage{pdfpages}
% if you are not using xelatex ou lualatex, replace by the encoding you are using
%\usepackage{CJKutf8}                              % if you need to use CJK to typeset your resume in Chinese, Japanese or Korean
\usepackage{multicol}

\usepackage{xcolor}
% adjust the page margins
\usepackage[scale=0.9,top=1.5cm, bottom=0.5cm]{geometry}
% \usepackage[scale=0.75]{geometry}
%\setlength{\hintscolumnwidth}{3cm}                % if you want to change the width of the column with the dates
%\setlength{\makecvtitlenamewidth}{10cm}           % for the 'classic' style, if you want to force the width allocated to your name and avoid line breaks. be careful though, the length is normally calculated to avoid any overlap with your personal info; use this at your own typographical risks...
\usepackage{xpatch}
\xpatchcmd\cventry{,}{}{}{}

% personal data

\name{Ivan}{Ogloblin}                               % optional, remove / comment the line if not wanted
\firstname{Ivan} % Your first name
\lastname{Ogloblin} % Your last name

% All information in this block is optional, comment out any lines you don't need
\title{Curriculum Vitae}

% \address{70 Absolute Ave.}{L4Z 0A4 Mississauga}{Canada}% optional, remove / comment the line if not wanted; the "postcode city" and and "country" arguments can be omitted or provided empty
\vspace*{3mm}
% optional, remove / comment the line if not wanted
% \phone[fixed]{+2~(345)~678~901}                    % optional, remove / comment the line if not wanted
% \phone[fax]{+3~(456)~789~012}                      % optional, remove / comment the line if not wanted
%  \homepage{linkedin.com/in/jondoe}                         % optional, remove / comment the line if not wanted
%\social[linkedin]{AlyaNovikova}
% \extrainfo{additional information}                 % optional, remove / comment the line if not wanted
%photo[64pt][0.4pt]{picture}                       % optional, remove / comment the line if not wanted; '64pt' is the height the picture must be resized to, 0.4pt is the thickness of the frame around it (put it to 0pt for no frame) and 'picture' is the name of the picture file
% \quote{Some quote}                                 % optional, remove / comment the line if not wanted

% to show numerical labels in the bibliography (default is to show no labels); only useful if you make citations in your resume
%\makeatletter
%\renewcommand*{\bibliographyitemlabel}{\@biblabel{\arabic{enumiv}}}
%\makeatother
%\renewcommand*{\bibliographyitemlabel}{[\arabic{enumiv}]}% CONSIDER REPLACING THE ABOVE BY THIS

%----------------------------------------------------------------------------------
%           footer
%----------------------------------------------------------------------------------
% bibliography with mutiple entries
%\usepackage{multibib}
%\newcites{book,misc}{{Books},{Others}}
%----------------------------------------------------------------------------------
%            content
%----------------------------------------------------------------------------------
%\makecvfooter
\begin{document}
	%\begin{CJK*}{UTF8}{gbsn}                          % to typeset your resume in Chinese using CJK
	%-----       resume       ---------------------------------------------------------
	\vspace*{-1.05mm}
	\makecvtitle
	\vspace*{-10mm}
	
	\section{Ausbildung}
	\cventry{}{}{Saint-Petersburg State University}{Sept 2019 - Juli 2023}{\hspace*{-2.5 mm} Bachelor-Abschluss in Informatik und Softwaretechnik }{}
	\cventry{}{}{Pontifical Catholic University of Rio de Janeiro}{Sept 2022 - Juli 2024}{\hspace*{-2.5 mm} Masterabschluss in Mathematik}{}
	{}{Verwandten Studienleistungen:}
	\vspace{-1.0em}\begin{small}
		\begin{multicols}{4}
			\begin{itemize}
				\item C++
				\item Kotlin
				\item Python
				\item Haskell
				\item Scala
				\item Algorithmen
				\item Parallele Programmierung
				\item Math logik
				\item Maschinelles Lernen
				\item Unix
				\item Betriebssysteme
				\item Algebra
				\item Mathematische Analyse
				\item Zufallsprozesstheorie
				\item Diskrete Mathematik
				\item Statistiken
				\item C\#
				\item Datenbanken
				\item Quanten-Computing
				\item Quanteninformationen
				\item JavaScript
				\item HTML and CSS
				\item Netzwerke
				\item Quantenmechanik 
			\end{itemize}
	\end{multicols}\end{small}
	
	\section{Arbeitserfahrung}
	
	\cventry{}{}{QC Design Berater}{Februar 2024 - aktuell\vspace{-1.0em}}{}{
		% Detailed achievements:%
		%\begin{itemize}
		Arbeiten Sie in einem Startup-Unternehmen an der Entwicklung von Software zur Simulation von Quantencomputermodellen in Python. Arbeitete an der Optimierung der Leistung des Codespace-Simulators, was das Verständnis des Algorithmus und die Implementierung von Optimierungen beinhaltete, die hauptsächlich auf der Manipulation von Datenstrukturen basierten. Wir arbeiten derzeit am Xpauli-Simulatorprojekt, um unsere Software zur schnellsten der Welt zu machen.
	}
	%\end{itemize}}
	\cventry{}{}{Huawei assistierender Ingenieur, Entwickler}{Oktober 2021 - Januar 2022\vspace{-1.0em}}{}{
		% Detailed achievements:%
		%\begin{itemize}
		Arbeitete am Backend C\#/.netASP/EntityFramework/Autofac + Frontend 3js/react/VR. Entwickeltes System zur Paketkommunikation ohne Verzögerung, das zwischen http- und signalR-Anfragen wechselt. Hat Forschungsarbeiten zur Handschrifterkennung mithilfe eines Faltungsnetzwerks unter Mensch-Computer-Interaktionen“ durchgeführt. Habe mich mit CNN-, RNN- und LSTM-Strukturen vertraut gemacht.
		%\end{itemize}
	}
	\cventry{}{}{Yandex Entwicklerpraktikant}{Juli - Sept 2021\vspace{-1.0em}}{}{
		% Detailed achievements:%
		%\begin{itemize}
		Arbeitete in zwei Teams am Backend C++/Python/SQL. Entwickeltes Unterstützungssystem für Trainingsskripte, um mit einer optimierten Struktur zum Speichern variabler Protokolle zu arbeiten. Schrieb Tests für Komponenten, die zur Aufbereitung von Daten für ein neuronales Netzwerk verwendet wurden, das Empfehlungen abgibt. Habe mich mit den Konzepten von Diensten und Hebeln vertraut gemacht. Tauchen Sie in die Feinheiten der Kommunikation zwischen Diensten und Systemen ein, um Informationen mit Fehlern zum Debuggen zu übertragen.
	}
	
	\section{Entwicklerprojekte}
	
	\cventry{}{}{Strawberry fields composer}{2023\vspace{-1.0em}}{}{
		% Detailed achievements:%
		Erstellung einer Website zur Simulation linearer und nichtlinearer Optik für Quantenberechnungsmodelle. Gebrauchte Django, Bootstrap und StrawberryFields. \href{https://strawberryfields.ai/}{\textcolor{blue}{Strawberry fields}} ist eine Basis für Forschungsrichtungen. Derzeit wird es auf einem externen kostenlosen Dienst gehostet. Bitte warten Sie, während es geladen wird! 
		\href{https://strawberryfieldscomposer.onrender.com}{(\textcolor{blue}{website})} (\href{https://github.com/StudioShader/SF_Composer}{\textcolor{blue}{github}})
	}
	
	\cventry{}{}{Archiver}{ 2019\vspace{-1.0em}}{}{
		% Detailed achievements:%
			 C++ Verwendete den Huffman-Algorithmus in der Implementierung zur Datenkomprimierung und -dekomprimierung \href{https://github.com/StudioShader/huffman-archiver}{(\textcolor{blue}{github})}}
	% \vspace{1.0em}
	
	%	\cventry{}{}{Multithreading Paint
		%	}
	%	{July 2018 \vspace{-1.0em}}{}{
		%		% Detailed achievements:%
		%		\begin{itemize} \textbf {
				%			\itemG\url{https://github.com/AlyaNovikova/Multithreading-Paint}
				%			\item Programm implemented in \textbf{Java} with the use of multithreading.
				%			\item Multithreading Paint is a drawing program that allows multiple users to sketch on the same canvas simultaneously.
				%	\end{itemize}}
		
		
		
		\cventry{}{}{Vacanter}{2019\vspace{-1.0em}}{}{
			% Detailed achievements:%
			Der Vacanter ist eine mobile Anwendung zum Matching von Arbeitgebern und potenziellen Arbeitnehmern. Ich habe eine Datenbank und ein Backend-System für die Anwendung bereitgestellt postgreSQL, python, Datagrip.
			\href{https://github.com/AndreyYurko/hhTinder}{(\textcolor{blue}{github})} 
		}
		
		\section{Frühere Erfolge}
		
		\cventry{}{}{ICPC}{2020\vspace{-1.0em}}{}{
			% Detailed achievements:%
			\begin{itemize}
				\item \href{https://github.com/StudioShader/StudioShader/blob/main/2019-Northwestern_Russia-PLACE.pdf}{41 Platz, Northwestern Russia Regional Contest St.Petersburg, Oktober 26, 2019}
				\item \href{https://github.com/StudioShader/StudioShader/blob/main/2020-Northwestern_Russia-PLACE.pdf}{Auszeichnung, Northwestern Russia Regional Contest St.Petersburg, 14 November, 2020}
				%			\item \href{https://diploma.rsr-olymp.ru/files/rsosh-diplomas-static/compiled-storage-2018/by-code/117292234832/color.pdf}{Top 174 out of 1404 in "Open olympiad in Mathematics" 2017}
				%			\item \href{https://diploma.rsr-olymp.ru/files/rsosh-diplomas-static/compiled-storage-2018/by-code/117272475400/color.pdf}{Top 109 out of 1103 in "Open olympiad in Physics" 2018}
		\end{itemize}}
		
%		\cventry{}{}{Open olympiad}{2018\vspace{-1.0em}}{}{
%			% Detailed achievements:%
%			\begin{itemize}
%				\item \href{https://diploma.rsr-olymp.ru/files/rsosh-diplomas-static/compiled-storage-2018/by-code/117272475400/color.pdf}{Top 60 out of 1100 in "Open olympiad in Mathematics" 2018 and 2016}
%				\item \href{https://diploma.rsr-olymp.ru/files/rsosh-diplomas-static/compiled-storage-2018/by-code/117292234832/color.pdf}{Top 174 out of 1404 in "Open olympiad in Mathematics" 2017}
%				\item \href{https://diploma.rsr-olymp.ru/files/rsosh-diplomas-static/compiled-storage-2018/by-code/117272475400/color.pdf}{Top 109 out of 1103 in "Open olympiad in Physics" 2018}
%		\end{itemize}}
		
%		\cventry{}{}{International scientific school conference "XVIII Kolmogorov Readings" }{2019\vspace{-1.0em}}{}{
%			% Detailed achievements:%
%			\begin{itemize}
%				\item I took \href{https://internat.msu.ru/media/uploads/2018/05/pobediteli-informatika-na-sajt.pdf}{\textcolor{blue}{third place}} in the discipline of computer science and mathematical modeling
%		\end{itemize}}
		
		\section{Programmierkenntnisse}
		\begin{itemize}
			\item C++, Python, C\#, C, Java, JavaScript, HTML, CSS, Kotlin, Haskell, Scala, SQL, Lean
			\item ASPnet, EntityFramework, Microsoft SQL Express, React, three.js, postgreSQL, Django, Bootstrap
			\item Git, Linux, Unity3D, SVN, Blender(3d modeling), protobuff, Shiny, Docker
			
		\end{itemize}
		%\vspace{6.0em}
		\newpage
		\section{Akademische Erfahrung}
	%	\cventry{}{}{Photonic quantum computer architecrute}{2023-2024\vspace{-1.0em}}{}{
		%	As a Master's thesis I am doing research on architecture of photonic quantum computers. Specifically I research fault-tolerant approaches using surface codes and fusions (FBQC) with photons encoding qubits in KLM protocol. During this work I wrote in detail how to construct a universal gate set in this setup and showed how to mitigate some type of errors and how to run full cycle of any quantum algorithm in this architecture. (text to be published)
	%	}
		\cventry{}{}{Bachelorarbeit}{2022-2023\vspace{-1.0em}}{}{
			Im Rahmen meiner Abschlussarbeit habe ich mithilfe genetischer Algorithmen optimale Schemata für verschränkte Transformationen in der linearen Quantenoptik erforscht. Ich habe einen genetischen Algorithmus mit GPUs auf Python (Pytorch) geschrieben, um nach neuen Verschränkungsschemata zu suchen. Es wurden neue Schemata zur Ermittlung des maximal verschränkten Zustands sowie zur Implementierung von CX-äquivalenten Gattern entwickelt. Obwohl es nicht möglich war, die Operationswahrscheinlichkeit zu verbessern, wurde die Hypothese aufgestellt, dass es in den Schemata im KLM-Protokoll unmöglich ist, ein Schema zu finden, das die Transformation implementiert, was kein perfekter Verschränkunger wäre oder zumindest nicht gleichwertig mit CX wäre. Sie können die Präsentation in
			\href{https://github.com/StudioShader/galopy/blob/master/slides(eng).pdf}{\textcolor{blue}{diesem Repository}}.
		}
	\cventry{}{}{Untersuchung der Auswirkung von Rauschen auf effiziente Quantensuchalgorithmen}{2022\vspace{-1.0em}}{}{
		Semesterprojekt zum Thema "Untersuchung der Auswirkung von Rauschen auf effiziente Quantensuchalgorithmen". In diesem Projekt sollte ich verbesserte Quantensuchalgorithmen für unstrukturierte Datenbanken implementieren. Sie basieren auf dem Grover-Algorithmus und werden \href{https://doi.org/10.1007/s11128-021-03165-2}{\textcolor{blue}{in diesem} Artikel beschrieben}. Gezeigt werden die Ergebnisse von Testalgorithmen für ein Problem mit nicht mehr als fünf Qubits. Meine Aufgabe bestand darin, die Grenzen von Quantensuchalgorithmen weiter zu erforschen. Zuerst habe ich mit Qiskit verbesserte Suchalgorithmen implementiert. Zweitens habe ich eine Umgebung zum Testen von Algorithmen mit unterschiedlichen Rauschmodellen und unterschiedlicher Anzahl von Qubits erstellt. Schließlich habe ich den Einfluss von Rauschen auf Variationen des Algorithmus untersucht. In meinen Experimenten habe ich ein Modell des thermischen Relaxationsrauschens und eine Kopplungskarte eines realen Geräts verwendet: "Melbourne". Dadurch verstand ich, wie man solche Experimente durchführt, um Schätzungen der Rauschparameter für einen realisierbaren Betrieb des Algorithmus zu erhalten. Details finden Sie in  \href{https://github.com/StudioShader/QPSA/blob/main/presentation.pdf}{\textcolor{blue}{diesem Artikel}} oder in \href{https://github.com/StudioShader/QPSA}{\textcolor{blue}{dieses Repository}}.
	}
		\cventry{}{}{Quantenalgorithmen für VRP- und VRPTW-Probleme}{2021\vspace{-1.0em}}{}{
			Ein Semesterprojekt zum Thema "Quantum Algorithms for VRP and VRPTW (Vehicle Routing Problem with Time Windows) Problems" mit Anwendung auf die realen Fallprobleme beim Bau der Routen für Bohrmaschinen für die Ölförderung in Zusammenarbeit mit GazpromNeft. Mir wurde direkt die Aufgabe übertragen, aktuelle Best Practices zur Lösung logistischer Probleme auf klassischen Computern zu studieren. Der nächste Schritt bestand darin, aktuelle Ergebnisse zur Lösung dieses Problems durch Quanten- und quanteninspirierte Algorithmen zu untersuchen. Ich habe eine Reduktion dieses Problems auf QUBO (Quadratic Unconstrained Binary Optimization) und eine Lösung unter Verwendung von Quantenoptimierungsalgorithmen wie VQE und QAOA gefunden. Dann sollte ich einen einfachen Löser für das Multi-Travelling-Salesman-Problem für kleinräumige Probleme (Spielzeugproblem, bis 7 Qubit) entwickeln. Es kann lokal auf einem Simulator mit Qiskit ausgeführt werden.\\(Leider kann ich aufgrund der Datenschutzbestimmungen von GazpromNeft keinen Code weitergeben)
		}
		\cventry{}{}{Lehrer-Assistent}{2023\vspace{-1.0em}}{}{
			Ich habe als Hilfslehrer für die Vorlesung "Einführung in die Quantenberechnung" gearbeitet und Hausaufgaben für Prof. \href{https://scholar.google.com/citations?user=UnZl40AAAAAJ&hl=en}{\textcolor{blue}{Sergey Tikhomirov}}. Zerlegte Shors Algorithmus in Unteraufgaben und lernte, Konzepte von den Grundlagen bis zum HHL zu erklären.
		}
%		\cventry{}{}{Courses}{2021-2022\vspace{-1.0em}}{}{
%			Almost all of these topics were understood by the book\\ \href{https://www.cambridge.org/ru/academic/subjects/physics/quantum-physics-quantum-information-and-quantum-computation/quantum-computation-and-quantum-information-10th-anniversary-edition?format=HB&isbn=9781107002173}{\textcolor{blue}{"Quantum Computation and Quantum Information"} Michael A. Nielsen, Isaac L. Chuang}
%			\begin{itemize}
%				\item Course on introduction to quantum computations: Grover's algorithm, Deutsch–Jozsa algorithm, quantum permutations, quantum Fourier Transform, quantum search, Q-RAM, Shor's algorithm.
%				\item Course on quantum information: density operator, noise in quantum systems, closeness of quantum states, quantum correction codes and their realization, classical and quantum entropy, bandwidth of quantum channels, transmission of quantum information over a noisy quantum channel, quantum cryptography.
%				\item Additional seminar with the `GazpromNeft` team: Phase estimation algorithm, QAOA algorithm, QAA algorithm, VQE algorithm, quantum search as quantum simulation, black box algorithm limits, speed up of NP-complete problems, quantum search optimality, quantum search in unstructured database, physical realization of quantum computer: harmonic oscillator, optical photon quantum computers, optical cavity quantum electrodynamics
%			\end{itemize}
%		}
		
		\cventry{}{}{Qiskit Global Summer School 2022 - Quantenexzellenz}{2022\vspace{-1.0em}}{}{
			Ich habe an der Qiskit Global Summer School 2022 teilgenommen und mich hervorgetan, die sich den Quantensimulationen widmete. Die Hauptaufgabe bestand darin, neue Wege zur Simulation des Hamilton-Operators für ein bestimmtes physikalisches System zu finden. Ich habe es erfolgreich gelöst und ein verdient \href{https://www.credly.com/badges/3304071b-2191-46fe-9de6-0b1cc019a06f/public_url}{\textcolor{blue}{Abzeichen on Credly}}.
		}
		\cventry{}{}{Quantencomputing und Quanteninformation mittels NMR}{2022\vspace{-1.0em}}{}{
			Ich habe an der 6. Fortgeschrittenenschule für Experimentalphysik des CBPF (Brasilianisches Zentrum für Physikforschung) teilgenommen und hervorragende Leistungen erbracht und einen Abschluss erworben  \href{https://github.com/StudioShader/StudioShader/blob/main/Certificados-21.pdf}{\textcolor{blue}{Zertifikat}}. Ich hatte Erfahrung mit der Bedienung eines echten NMR-Geräts und der Durchführung von Experimenten zur Kodierung und Verschränkung zweier Qubits.
		}
		
		
		
		% \vspace{-1.0em}
		
		% \vspace{1.0em}
		
		\section{Sprachen}
		Russisch (Muttersprache), Englisch (fließend), Portugiesisch (sprechend)
		
		\vspace*{\fill}
		\name{}{}
		\title{}
		\phone[mobile]{+55~(21)~98276-0687}
		\address{R. João Borges, 240 - casa 22}{Alto da Boa Vista, Rio de Janeiro}{ Brazil }
		\email{ivanogloblin2022@gmail.com}
		\social[github]{StudioShader}   
		\makecvtitle
		%\section{Languages}
		%\begin{itemize}
		%	\item Russian (Native), English (Upper-Intermediate)
		%\end{itemize}
		%\makecvfoot
	\end{document}